
\chapter{Conclusão}

Com os resultados encontrados, é possível perceber que, apesar de algumas poucas diferenças a amostra deste trabalho se comportou de forma condizente àquela encontrada no artigo base \cite{empiricalStudy2016}. Desde de o ranking de utilização de cada prática até as barreiras enfrentadas pelos desenvolvedores, houve uma congruência razoável entre os dois resultados. 

Sobre o processo de integração contínua, pode-se inferir que as técnicas ligadas ao \emph{merge} contínuo estão pouco presentes, apesar de \emph{Developer Awareness} ser a prática mais utilizada da amostra. A maioria entrevistada integra o código na \emph{branch} principal apenas no final da \emph{sprint}.

É possível perceber ainda que todas as práticas ligadas à \emph{Deployment} contínuo estão muito presentes na amostra. Mesmo com a prática de \emph{Developer on Call} sendo mais utilizada de forma implícita do que de fato definida, e com algumas empresas interpretando-a como uma má prática, ela está em 4º lugar no ranking de utilização montado e é a de menor colocação do grupo.

Por fim, foi possível perceber que a amostra utiliza muito pouco as práticas de entregas parciais. Foi possível identificar técnicas desconhecidas por um grupo razoável de entrevistados - \emph{Dark Launches} - e até técnicas que não foram utilizados por nenhum dos participantes - Testes A/B.

\section{Ameaças a validade}

Mesmo com a metodologia definida e seguida utilizando métodos conhecidos pela comunidade científica, é necessário levantar possíveis ameaças à validade dos resultados encontrados neste trabalho. Uma ameaça plausível é a quantidade relativamente pequena de entrevistas feitas.

É importante salientar também que os códigos gerados durante o processo de codificação das entrevistas sofreram um certo enviesamento visto que este trabalho é uma replicação de um estudo, então o autor tinha em mente que assuntos estavam sendo procurados na fala durante o levantamento de códigos. 

Outra ameça que deve ser levada em conta é o processo de \emph{coding} realizado. Ele foi feito baseando-se no áudio das entrevistas, e não nos texto transcritos, como geralmente é feito \cite{groundedTheory}. Isso pode tornar os códigos enviesados ou até mesmo significar a falta de códigos importantes que poderiam ter sido levantados pelo processo original.


\section{Trabalhos Futuros}

Como trabalhos futuros, pode ser feito o mesmo estilo de entrevistas com um grupo maior de pessoas. A reaplicação poderia levantar novas barreiras e até identificar lacunas que não puderam ser vistas com a amostra deste trabalho. Outro ponto que também seria de grande valia é a pesquisa sobre o mesmo tema, mas de forma quantitativa, com o objetivo de entender melhor que práticas estão sendo utilizadas na indústria de Recife com um grupo maior de entrevistados. 
