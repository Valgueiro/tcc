
\section{Introdução}

As práticas de integração, entrega e implantação contínuos \cite{fowlerCI, fowlerCD} --- originalmente conhecidas pelos termos em inglês \emph{Continuous Integration} (CI), \emph{Continuous Delivery} e \emph{Continuous Deployment} (CD) \footnote{Neste trabalho, assim como no estudo original, a sigla CD se refere às práticas de entrega e implantação contínuas}, respectivamente --- já são muito difundidas e utilizadas por empresas de tecnologia em todo o mundo. Segundo o mais recente relatório anual do estado das metodologias ágeis \cite{stateAgileReport2020}, 55\% dos participantes reportaram que sua organização utiliza a técnica de integração contínua. Também nesta mesma pesquisa foi encontrado que 41\% utilizam a técnica de entrega contínua e 36\%, implantação.

Estes números elevados são causados principalmente pelos benefícios que a utilização destas técnicas trazem para a equipe e para o produto em desenvolvimento. Estudos comprovam que os desenvolvedores envolvidos se sentem mais produtivos quando usam as práticas de CI/CD \cite{hilton2016} e o seu uso traz maior qualidade ao software que está sendo produzido \cite{savor2015}. 

É possível, no entanto, perceber uma grande falta de estudos a respeito de como essas práticas foram importadas para as empresas ao redor do mundo \cite{empiricalStudy2016} -- as pesquisas são geralmente voltadas apenas para as corporações globais como Facebook \cite{savor2015} e Google \cite{googleCi}. Como pode haver bastante variação no nível de adoção de CI, CDE e CD por localização e tipo de empresa, é importante entender qual o nível de adoção para um determinado ecossistema específico de software, principalmente para saber que causas justificam o nível de adoção encontrado, e que ações são necessárias para aumentá-lo. Nesse sentido, neste trabalho investigamos  a adoção no ecossistema do Porto Digital, em Recife; a capital pernambucana é sede de um dos grandes pólos tecnológicos do Brasil: cerca de 330 empresas e 11 mil trabalhadores, com faturamento anual de R\$ 2,3 bilhões em 2019 \cite{portoDigital}.

Um estudo a respeito de como as técnicas de CI/CD foram adotadas na indústria de Recife serve como ponto de partida para pesquisas relacionadas ao levantamento de dores sentidas pelos desenvolvedores locais que inibem a utilização destas técnicas.

Neste contexto, este trabalho tem como objetivo entender, através de uma pesquisa qualitativa, quais as práticas e técnicas subjacentes de integração, entrega e implantação contínua adentraram nas empresas de Recife. Não obstante, o trabalho deseja descobrir que princípios e práticas subjacentes governam a adoção destas técnicas.

Dentre os achados, descobriu-se que a amostra não segue o \emph{Stairway to Heaven}\cite{empiricalStudy2016} proposto pelo estudo original. Em relação as práticas usadas pelos entrevistados, observou-se que a maioria não integra código numa frequência diária na \emph{branch} principal, indicando que a integração contínua não se faz muito presente no dia a dia das empresas da amostra. Quanto as práticas de implantação contínua, todas tiveram presença na amostra. Adicionalmente, a prática de testes A/B não teve relatos de sua utilização por nenhum entrevistado.
