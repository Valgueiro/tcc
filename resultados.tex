
\chapter{Resultados}

Neste capítulo serão apresentados os resultados encontrados com as entrevistas baseado na análises de dados apresentada no capítulo anterior. Primeiramente será focado o estudo do predomínio das práticas em Recife, seguido da análise de coerência do stairway to heaven. Por fim, serão apresentados princípios e práticas subjacentes que governam a adoção de CI/CD na amostra.

\section{Estudo de Predomínio das Práticas}

Com o intuito de responder a pergunta de pesquisa RQ1 - \emph{Quais as práticas de CI/CD são utilizadas pelas empresas em Recife?} - foi montada a tabela T3, presente na \figref{tabela_t3}, demonstrando a utilização de cada uma das práticas definidas para cada um dos participantes. Nesta, é demonstrado através da coloração da célula o grau de utilização de determinada prática por determinado participante. Assim, o verde mais escuro significa que o participante utiliza totalmente, enquanto o verde claro significa utilização parcial e, por fim, o branco denota a não utilização. A ideia desta tabela é replicar a Tabela 1 do artigo base, presente na \figref{tabela_1_artigo_base}.

\begin{figure}[ht]
\begin{center}
\includegraphics[scale=0.15]{tabelaT3.png}
\end{center}
\caption[Tabela T3]{
    Tabela T3 com o nível de utilização de cada uma das práticas, com as colunas ordenadas em ordem decrescente de uso.
}\label{tabela_t3}
\end{figure}

\begin{figure}[ht]
\begin{center}
\includegraphics[scale=0.5]{tabela1-artigo-base.png}
\end{center}
\caption[Tabela 1 do artigo base]{
    Utilização das práticas pelos participantes do artigo base.
    Fonte: Schermman et al (2016)
}\label{tabela_1_artigo_base}
\end{figure}

Com a tabela T3 (\figref{tabela_t3}) é possível perceber que \emph{Developer awareness [AWA]} foi a prática mais encontrada em toda a amostra, sendo, no mínimo, parcialmente utilizada por todos os entrevistados. Logo após, pode-se encontrar as práticas de \emph{Deployment Pipeline [PIP]} e \emph{Health Check [HC]} na segunda e na terceira colocação, respectivamente. No geral, também pode-se inferir que as técnicas de \emph{Partial Rollouts} são ainda muito pouco utilizadas, com as 3 práticas do grupo entre as quatro últimas colocadas. Vale a pena citar que nenhum dos entrevistados utiliza, mesmo que precariamente, a técnica de Testes A/B [AB].

Quando comparamos a Tabela T3 (\figref{tabela_t3}) com a tabela 1 do artigo base (\figref{tabela_1_artigo_base}), podemos perceber que há diferenças de posição entre práticas, mas que não há mudanças exorbitantes. Com a ajuda da \figref{diferenca_entre_posicoes_fig}, é possível perceber que há uma diferença de no máximo 2 posições na ordem de predomínio, ao comparar os resultados obtidos neste trabalho com os do artigo base. 

\begin{figure}[ht]
\begin{center}
\includegraphics[scale=0.15]{diferenca_entre_posicoes.png}
\end{center}
\caption[Diferença entre a ordem de predomínio das práticas]{
    Diferença entre o artigo base e este trabalho a respeito da ordem de predomínio das práticas
}\label{diferenca_entre_posicoes_fig}
\end{figure}

\section{Stairway to heaven}

Com o objetivo de responder a pergunta de pesquisa RQ2 - \emph{ O cenário de CI/CD nas empresas em Recife segue o "stairway to heaven" proposto no artigo?} - foi produzida a Tabela T2, presente na \figref{tabela_t2}. Esta contém a visualização de utilização das práticas, utilizando a mesma técnica de cores aplicada na Tabela T3 (\figref{tabela_t3})  e explicada na seção anterior, mas com as colunas ordenadas pela escada definida no \emph{Stairway to Heaven}.

\begin{figure}[ht]
\begin{center}
\includegraphics[scale=0.15]{tabelaT2.png}
\end{center}
\caption[Tabela T2]{
    Tabela T3 com o nível de utilização de cada uma das práticas, com as colunas ordenadas na ordem do \emph{Stairway to Heaven}.
}\label{tabela_t2}
\end{figure}

Com a Tabela T2 (\figref{tabela_t2}) é possível perceber que a amostra deste estudo também não segue a evolução proposta pelos autores do artigo base \cite{empiricalStudy2016}. Isso fica claro quando percebe-se que não há, em nenhum dos entrevistados, uma relação clara entre a coloração da coluna com a sua anterior. É possível notar também que há um vão nas duas primeiras práticas, seguido de uma grande utilização da terceira, confirmando a tese de que a \emph{Stairway to Heaven} não é coerente na amostra. É interessante destacar que a amostra do próprio artigo base também tem a mesma falha de coerência.

\section{Analise das Práticas}

Nesta seção será abordado, para cada uma das práticas, as principais curiosidades encontradas na amostra. Esta tem como objetivo responder a pergunta RQ3: \emph{Quais são os princípios e práticas subjacentes que governam a adoção de CD na indústria?}. As seguintes subseções agrupam as práticas marginais através das práticas gerais definidas no \emph{Stairway to Heaven}, inclusive seguindo a ordem deste.

Para poder definir padrões encontrados na amostra o autor, após a fase de agrupamento semântico, juntou os todas as categorias e super categorias ligadas a cada uma das práticas baseando-se principalmente na nota de utilização. Com a leitura e reflexão deste conjunto, o autor procurou por princípios que governam a adoção ou não de de cada prática. Foi feito também um estudo comparativo com os resultados obtidos pelo artigo base \cite{empiricalStudy2016} para analisar discrepâncias e congruências entre os dois contextos de estudo.

\subsection{Integração Contínua}

Nesta subseção, serão abordadas as práticas ligadas ao processo de Integração contínua: \emph{Trunk Based Development} [TBD], \emph{Feature Toggles} [FT] e \emph{Developer Awareness} [AWA].
\subsubsection{Trunk Based Development [TBD]}
Na amostra é possível perceber que a técnica de \emph{Trunk Based Development} [TBD] não é tão amplamente adotado, visto que apenas 4 entrevistados utilizam ao menos parcialmente. Destes, 2 comentam que entregam novas versões aos clientes baseados em novas funcionalidades, e não em \emph{sprints}. Já entre os que não utilizam, 5 integram o código apenas no final da \emph{sprint}. Deste grupo, 2 utilizam a metodologia \emph{Git Flow} \cite{gitFlow}, que define uma maneira de manusear várias branches ao mesmo tempo de modo que os desenvolvedores deparem com o mínimo de conflitos possível e que software seja entregue em versões bem definidas.

\subsubsection{Feature Toggles [FT]}

No estudo foi possível perceber que, na amostra, a prática de \emph{Feature Toggles} [FT] é raramente utilizada, assim como no artigo base \cite{empiricalStudy2016}. Esta técnica foi encontrada apenas na equipe do participante P5, que a utilizava para esconder funcionalidades enquanto testes manuais ainda estavam sendo feitos. É interessante notar que este participante também foi um dos poucos que utilizava a técnica de \emph{Canary Releases} [CAN].

Entre o grupo dos que não utilizavam, 7 só enviam código novo para produção quando a funcionalidade está concluída. Deste grupo, 2 comentaram que fazem uso de conceito de épicos, onde uma história de usuário se prolonga por mais do que apenas uma sprint. É também interessante notar que P3 está em vias de utilizar esta técnica para reduzir conflitos de \emph{merge} e \emph{rebase} devido ao grande número de desenvolvedores em seu time, como podemos ver na citação:

\begin{quotation}[]{P3 - Web - CORP}
    "O meu time tem 23 [pessoas]. [...] a gente tá trabalhando em funcionalidades muito distintas, então às vezes acontece de termos um paralelismo de branches muito grande. [...] é muito complicado ‘mergear’ e fazer rebase de tudo. Realmente dá muitos conflitos"
\end{quotation}


\subsubsection{Developer Awareness [AWA]}

Na amostra é possível identificar que a prática de \emph{Developer Awareness} [AWA] é amplamente adotado nas companhias, assim como na amostra do artigo base \cite{empiricalStudy2016}. Em 6 entrevistas foi possível perceber que o time de desenvolvimento era o mesmo responsável pela entrega e manutenção da aplicação. Jà outras 2, há um time específico de \emph{DevOps}, mas não havia grandes silos entre este e a equipe de desenvolvimento. Um caso interessante foi o de P11, que, apesar de um conhecimento espalhado dentro da equipe, há receio e insegurança por parte de alguns a respeito de questões de infraestrutura no geral.


\subsection{Deployment Contínuo}

Nesta subseção, serão abordadas as práticas ligadas ao processo de \emph{Deployment} contínuo: \emph{Health Checks} [HC], \emph{Developer on Call} [DOC] e \emph{Deployment Pipeline} [PIP].

\subsubsection{Health Checks [HC]}

Sobre a prática de \emph{Health Checks} [HC], é possível perceber que, apesar de não ser o mais adotado - como foi no artigo base, ainda tem destaque entre as outras, presente na segunda colocação. Na amostra, 7 entrevistados continham pelo menos uma forma rudimentar de checagem e alertas, e destes, 2 continham apenas verificações não muito complexas. Um ponto interessante que surgiu foi o fato de P4 achar que não era necessário utilizar esta técnica por estar em fase de prototipação.

\subsubsection{Developer on Call [DOC]}

Na amostra é possível perceber que a técnica de \emph{Developer on Call} [DOC] é mais adotada de forma implícita do que de fato definida - 3 entrevistados estão em times que funcionam desta forma. Contudo, 5 pessoas do grupo não utilizam esta prática: alguns comentaram que a confiança nos testes automáticos faz com que não utilizem a prática, enquanto outro comentou que a prática é inclusive mal vista pela empresa.

Uma dicotomia interessante foi encontrada entre os resultados da amostra deste trabalho e o do artigo base \cite{empiricalStudy2016}. Este último comenta que a prática já esta sendo largamente aceita nas organizações atualmente, e inclusive um dos entrevistados comenta que essa responsabilidade de ficar até mais tarde para resolver problemas leva os desenvolvedores a escrever e testar seus códigos mais veemente. Tal argumentação tem como base a seguinte citação retirada do artigo e traduzida:

\begin{quotation}[]{P14 (do artigo base) - Web - CORP}
    "Se você não tem testes suficientes e faz deploy de um código ruim isso vai se voltar contra você pois você estará de plantão e terá que dar suporte a isto."
\end{quotation}

Contudo, com a seguinte citação de P2, é possível perceber que há um sentimento contrário: confia-se no processo de qualidade e, por isso, não necessitam de plantão.

\begin{quotation}[]{P2 - Web - CORP}
    "Temos o ciclo de QA, se encontrar alguma coisa a gente vê […] não precisa isso de plantão não." 
\end{quotation}

\subsubsection{Deployment Pipeline [PIP]}

Em relação à prática de \emph{Deployment Pipeline} [PIP] é possível inferir que ela é amplamente adotada pela amostra, visto que apenas 2 entrevistados obtiveram nota 0 (não utiliza). No artigo base \cite{empiricalStudy2016} é possível perceber um resultado semelhante a este. Uma grande parte dos entrevistados segue o mesmo padrão para qualquer tamanho da mudança. Outros 2 tinham alguns processos automatizados, mas a \emph{pipeline} era diferente dependendo do tamanho da mudança.

É legal perceber ainda que alguns times ainda pecam na falta de automação dos processos da \emph{pipeline}. Isto pode ser em decorrência da complexidade da automação, devido às tecnologias utilizadas, ou da falta de prioridade do time para tal.
 
\subsection{Entregas Parciais}

Nesta subseção, serão abordadas as práticas ligadas ao processo de Entregas Parciais: \emph{Canary Releases} [CAN], \emph{Dark Launches} [DAR] e Testes A/B [AB]. Na amostra foi possível perceber que todas as técnicas são muito pouco utilizadas: todas estão entre as 4 últimas colocações. É valido também notar que as 3 mantiveram a mesma ordem de utilização do processo do artigo base \cite{empiricalStudy2016}.

\subsubsection{Canary Releases [CAN]}

A técnica de \emph{Canary Releases} [CAN] apareceu nos contextos de jogos e no de sistemas embarcados. Na equipe do entrevistado P5, que trabalha no domínio de jogos eletrônicos, os principais jogadores - conhecidos como "baleias" - são escolhidos para participar de um \emph{early access} de novas funcionalidades.  Esses testes levam em torno de 1 semana. 

Já na equipe de P4, que trabalha com sistemas embarcados, a prática era necessária devido ao contexto de atuação e às tecnologias utilizadas. Como o sistema que está em fase de prototipação servirá para o contexto médico, vários testes de campo deveriam ser feitos para garantir que todas as funcionalidades estivessem de acordo com o esperado. Para os testes, o cliente que contratou a empresa de P4 escolhia a quantidade de pessoas e local que serviria como validação de funcionalidades.

Entre os entrevistados qe não utilizam a prática de \emph{Canary Releases}, 3 têm um ambiente de homologação para testes e validação de requisitos, mas este utiliza dados diferentes dos de produção. Outros 3 comentam que as features são sempre entregues para todos os usuários ao mesmo tempo.


\subsubsection{Dark Launches [DAR]}

Do grupo das práticas associadas a Entregas Parciais, \emph{Dark Launches} [DAR] foi a segunda mais utilizada, mas a menos conhecida entre os entrevistados. Isto se confirma com o fato de que 6 entrevistados disseram nunca ter utilizado, e 3 destes disseram especificamente que não conheciam a técnica. Importante notar que no estudo base \cite{empiricalStudy2016} esta é a menos utilizada do grupo de práticas.

\emph{Dark Launches} foi utilizado totalmente por P5 no contexto de jogos para testes manuais, e apenas parcialmente no contexto de WEB por P9 para validações de alguns cenários que dependiam de dados de produção.


\begin{quotation}[]{P5 - Jogos - Startup}
    "A gente implementa a funcionalidade mas condiciona a não aparecer para o usuário até que a gente queira."
\end{quotation}

\subsubsection{Testes A/B [AB]}

A prática de Testes A/B [AB] não foi identificada em nenhum dos participantes. Entre as principais causas levantadas para tal, 4 entrevistados comentaram que não utilizam pela baixa quantidade de usuários ativos no sistema. Outros 2 comentaram que a técnica não se aplicava ao contexto da aplicação. É interessante notar que esses dois motivos estão presentes no artigo base \cite{empiricalStudy2016}, contudo, ao contrário deste, ninguém na amostra comentou sobre problemas na arquitetura como causa para não utilização.


\begin{quotation}[]{P2 - WEB - Corp}
    "O sistema da gente - apesar de lidar com uma massa de dados muito grande - não têm tantos usuários, então não faz muito sentido [utilizar testes A/B]...."
\end{quotation}

Outro motivo importante foi levantado por P8, que comenta que aparentemente não existe na empresa o interesse em investir nesta prática. P3 comenta ainda que o time está mais focado em entregar novas funcionalidades, pois a demanda por parte do cliente está muito grande.

\section{Descobertas adicionais}

Esta subseção abordará sobre os dois tópicos marginais que foram levantados durante a análise dos dados obtidos pelas entrevistas: \emph{Code Review} e testes automáticos.

\subsection{Code Review}

A prática de \emph{Code Review} é bem vista pela maioria dos entrevistados, considerado uma técnica importante e essencial para o controle de qualidade de código. As razões para o uso são diversas, desde de impor boas práticas - por P7 - até o compartilhamento e nivelamento do conhecimento - por P11.

\begin{quotation}[]{P2 - WEB - Corp}
    "... a partir do momento que o sistema vai crescendo, o próprio desenvolvedor não tem a noção de que aquela sua mudança não é a melhor forma de fazer e que pode quebrar outras partes do sistema..."
\end{quotation}

Ainda na amostra, 3 entrevistados acreditam que a prática funciona principalmente para troca de conhecimento. Sobre isso, a entrevistada P10 comenta que agrega muito valor para ela como novata no time, mas acha que adiciona um tempo desnecessário na entrega de funcionalidades por pessoas mais experiente, visto que - para ela - a técnica não faria sentido neste caso.

Outro ponto importante foi a distinção entre dois relatos a respeito do engajamento do time com a prática. Enquanto P3 comenta que a equipe dela é bem aberta ao debate e está engajada com o processo, P7 fala que o processo funciona "mais ou menos", dependendo do humor dos revisores.

\subsection{Testes Automáticos}

Testes automáticos são, no geral, bem vistos e extremamente recomendados pela maioria dos entrevistados como forma de prevenir erros em tempo de execução. Contudo, o participante P2 comenta que ainda é contrário a depender somente deles como garantia de qualidade.

\begin{quotation}[]{P2 - WEB - Corp}
    "... automação [de testes] não resolve todos os problemas [relacionados a garantia de qualidade], ele vai identificar muita coisa, mas tem várias outras que precisamos do olhar de um testador…"
\end{quotation}

Na amostra, 5 entrevistados acreditam que o aumento da cobertura de testes automáticos pode diminuir a quantidade de bugs em produção. Dentro desse grupo, o participante P9 comenta que isto poderia, no entanto, diminuir a velocidade de entregas do time. P5 adicionou ainda que sente que a \emph{sprint} é muito curta e não consegue tempo dentro destas para adicionar testes automáticos.
