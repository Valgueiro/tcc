\resumo

As práticas de \emph{Continuous Integration}, \emph{Continuous Delivery} e \emph{Continuous Deployment} já estão presentes no cotidiano de grandes empresas de todo o mundo. E isto é devido, entre outras coisas, aos comprovados benefícios que a utilização destas técnicas trazem para a equipe e para o produto em desenvolvimento. Mesmo assim, poucos são os estudos que buscam a respeito do estado da arte destas práticas na maioria das empresas, principalmente no contexto da cidade de Recife.  Com isso, o presente trabalho objetiva a realização de uma pesquisa qualitativa para entender como as técnicas de integração, entrega e \emph{deployment} continuos foram importadas para as empresas recifenses, assim como identificar princípios e práticas adjacentes que governam a adoção destas técnicas. Foi descoberto que a maioria entrevistada integra o código de uma nova funcionalidade para a \emph{branch} principal apenas no final da \emph{Sprint}, não diariamente. Além disso, as práticas de entregas parciais são utilizados apenas por uma pouca parcela de desenvolvedores.

% Palavras-chave do resumo em Português
\begin{keywords}
    Integração Contínua, Deployment Contínuo, Recife, Engenharia de Software
\end{keywords}
