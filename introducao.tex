
\section{Introdução}

As práticas de integração, entrega e implantação contínuos \cite{fowlerCI, fowlerCD} (\emph{Continuous Integration} [CI], \emph{Continuous Delivery} e \emph{Continuous Deployment} [CD]) já são muito difundidas e utilizadas por empresas de tecnologia em todo o mundo. Segundo a pesquisa da empresa \emph{digital.ia}\cite{stateAgileReport2020}, 55\% dos participantes reportaram que sua organização utiliza a técnica de integração contínua. Também nesta mesma pesquisa foi encontrado que 41\% utilizam a técnica de entrega contínua e 36\%, implantação.

Estes números elevados são causados principalmente pelos benefícios que a utilização destas técnicas trazem para a equipe e para o produto em desenvolvimento. Estudos comprovam que os desenvolvedores envolvidos se sentem mais produtivos quando usam as práticas de CI/CD \cite{hilton2016} e o seu uso traz maior qualidade ao software que está sendo produzido \cite{savor2015}. 

É possível, no entanto, perceber uma grande falta de estudos a respeito de como essas práticas foram importadas para as empresas de todo o mundo \cite{empiricalStudy2016}, incluindo Recife -- as pesquisas são geralmente voltadas apenas para as corporações globais como Facebook \cite{savor2015} e Google \cite{googleCi}. A capital pernambucana é um dos grandes pólos tecnológicos do Brasil: somente no Porto Digital, localizado na cidade, há cerca de 330 empresas e 11 mil trabalhadores, com faturamento anual de R\$ 2,3 bilhões em 2019 \cite{portoDigital}.

Um estudo a respeito de como as técnicas de CI/CD migraram para a indústria de Recife serve como ponto de partida para pesquisas relacionadas ao levantamento de dores sentidas pelos desenvolvedores locais que inibem a utilização destas técnicas.

Neste contexto, este trabalho tem como objetivo entender, através de uma pesquisa qualitativa, quais as práticas e técnicas subjacentes de integração, entrega e \emph{deployment} contínuos adentraram nas empresas de Recife. Não obstante, o trabalho deseja descobrir que princípios e práticas subjacentes governam a adoção destas técnicas.

