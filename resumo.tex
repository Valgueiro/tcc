\resumo

As práticas de \emph{Continuous Integration} [CI], \emph{Continuous Delivery} e \emph{Continuous Deployment} [CD] já estão presentes no cotidiano de grandes empresas de todo o mundo. E isto é devido, entre outras coisas, aos comprovados benefícios que a utilização destas técnicas trazem para a equipe e para o produto em desenvolvimento. Mesmo assim, poucos são os estudos que investigam o estado da arte destas práticas na maioria das empresas, principalmente no contexto da cidade de Recife.  Com isso, o presente trabalho objetiva entender como as técnicas de integração, entrega e implantação contínuas foram importadas para as empresas recifenses, assim como identificar princípios e práticas adjacentes que governam a adoção destas técnicas. Para tanto, foi realizada uma pesquisa qualitativa por meio de entrevistas com 11 desenvolvedores de software de empresas de tecnologia sediadas na cidade do Recife. Foi descoberto que a amostra não segue o \emph{Stairway to Heaven}, teoria definida pelo artigo base deste trabalho e que sugere que a adoção deve seguir uma sequência específica de práticas. Não obstante, a maioria entrevistada integra o código de uma nova funcionalidade para a \emph{branch} principal apenas no final da \emph{Sprint}, não diariamente, o que não é consistente com as definições mais comuns de CI. Além disso, as práticas de entregas parciais são utilizadas apenas por uma pequena parcela de desenvolvedores.

% Palavras-chave do resumo em Português
\begin{keywords}
    Integração Contínua, Deployment Contínuo, Desenvolvimento Colaborativo, Engenharia de Software
\end{keywords}
