
\chapter{Introdução}

* Falar um pouco sobre técnicas de CI/CD
* Motivar o leitor
* Metodologia
* um pouco dos resultados
* Restante da estrutura do trabalho

[WIP - Retirado da proposta de TG]

As práticas de integração, entrega e implantação continuas (\emph{Continuous Integration, Continuous Delivery} e \emph{Continuous Deployment}) já são muito difundidas e utilizadas por empresas de tecnologia em todo o mundo. Segundo \cite{stateAgileReport2020}, 95\% dos participantes reportaram que sua organização pratica metodologias ágeis e 55\% utiliza a técnica de integração continua. Também nesta mesma pesquisa foi encontrado que 41\% utilizam a técnica de \emph{Continuous Delivery}, e 36\%, \emph{Continuous Deployment}.

Estes números elevados são causados principalmente pelos benefícios que a utilização destas técnicas trazem para a equipe e para o produto em desenvolvimento. Estudos comprovam que os desenvolvedores envolvidos se sentem mais produtivos quando usam as práticas ágeis \cite{hilton2016} e o seu uso traz maior qualidade ao software que está sendo produzido \cite{savor2015}. 

Ainda há, no entanto, uma grande falta de estudos a respeito de como essas práticas se exportaram para as grandes empresas de Recife \cite{empiricalStudy2016}, um dos grandes polos tecnológicos do Brasil. Somente no Porto Digital, localizado na capital pernambucana, há cerca de 330 empresas e 11 mil trabalhadores com faturamento anual de R\$ 2,3 bilhões em 2019 \cite{portoDigital}.

Percebe-se também uma quantidade notável de indivíduos equivocados a respeito dos conceitos e do uso das práticas supracitadas \cite{debbiche2014challenges}. O estudo \cite{citheater2019} identificou que 60\% dos projetos não utilizam a técnica de \emph{commits} contínuos, mesmo utilizando uma ferramenta que tem como objetivo implantar a prática de integração contínua. Os autores chamam este fenômeno de \emph{Continuous Integration Theather} (Teatro da integração contínua), e comentam que este processo produz um ambiente não saudável de desenvolvimento ágil.

Neste contexto, este trabalho tem como objetivo entender melhor, através de uma pesquisa quantitativa, como as práticas de integração, entrega e implantação continuas foram implantadas no âmbito das empresas de Recife, assim como avaliar se os termos utilizados são corretamente compreendidos pelos desenvolvedores destas, procurando assim possíveis ''Teatros'' recifenses. 

\section{Estrutura do Trabalho}
 O trabalho está divido em 6 capítulos, segmentados da seguinte forma:

 \begin{itemize}
     \item O Cápitulo 2 contém a Fundamentação Teórica deste trabalho, abordando sobre as práticas de CI/CD e definindo termos que serão utilizados durante o trabalho, além de alguns trabalhos relacionados
     \item O Capitulo 3 trás a motivação por trás deste trabalho
     \item O Capítulo 4 contém a metodologia utilizada para a montagem e execução das entrevistas, além de todos os passos da análise de dados
     \item No Capítulo 5 encontra-se os resultados obtidos pela entrevista qualitativa
     \item O Capitulo 6 contém a conclusão com uma análise final e perspectivas de trabalhos futuros.
 \end{itemize}
