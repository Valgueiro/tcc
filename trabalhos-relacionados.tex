\section{Trabalhos Relacionados}

Há vários trabalhos relacionados a benefícios e barreiras que existem na adoção de integração contínua em contextos variados. É possível perceber que no geral os desenvolvedores gostam de utilizar CI por garantir um desenvolvimento de uma forma mais segura e confiável \cite{googleCi} e por se sentirem mais produtivos \cite{hilton2016}. É possível perceber também que os desenvolvedores ainda acham que as ferramentas de CI são complicadas e difíceis de configurar \cite{hilton2016}. 

É interessante levantar ainda no contexto de integração contínua que utilizar as ferramentas voltadas para esta técnica sem adequar a cultura de desenvolvimento para tal leva a práticas não saudáveis de desenvolvimento, tais como \emph{builds} que levam muito tempo para concluir ou permanecem quebrados por longos períodos de tempo  \cite{citheater2019}. Outro estudo encontrou que a utilização da técnica apresenta um \emph{trade-off} entre velocidade e certeza no que diz respeito a garantia de qualidade do software \cite{hilton2016}.

Já a respeito de implantação contínua, foi encontrada a visão dos desenvolvedores do Facebook e de OANDA\footnote{OANDA é uma pequena empresa privada que provê informações de moeda e troca de moeda como um serviço. Diariamente os serviços da empresa movimentam bilhões de dólares.} a respeito desta técnica \cite{savor2015}. O artigo mostra que os desenvolvedores preferem \emph{deployments} mais rápidos por gerar maior qualidade de software e maior produtividade. Contudo, eles acreditam que há uma instabilidade maior e é uma metodologia inviável para sistemas críticos.

Traçando um comparativo com o estudo original \cite{empiricalStudy2016}, para a construção do questionário utilizado nesse estudo, houve uma adaptação das perguntas 1 e 2 do questionário original para que as questões 1 e 3 fossem elaboradas. Adicionalmente, foi proposta uma nova pergunta de pesquisa (PP2) para que se possa tentar entender se a \emph{stairway to heaven} proposta é seguida no contexto de Recife. No que diz respeito aos resultados, o que foi obtido com a amostra deste trabalho se comportou de forma condizente àquela encontrada no estudo original \cite{empiricalStudy2016}. Tanto no ranking de utilização de cada prática, bem como nas barreiras enfrentadas pelos desenvolvedores, houve uma congruência razoável entre os dois resultados, com apenas algumas diferenças no que tange a permutação de alguns elementos entre si. Isto serve como indicador de que tais práticas encontraram seu lugar na indústria e estão sendo implementadas de forma eficiente. 