
\chapter{Metodologia}
[WIP - retirado do primeiro draft do TCC]

Para o estudo, foi realizada uma pesquisa qualitativa através de entrevistas semi-estruturadas utilizando como base a entrevista utilizada no artigo base. 


\section{Pré-estudo}

Primeiramente, os dois entrevistadores discutiram a respeito do artigo base: An empirical study on principles and practices of continuous delivery and deployment [1]. Na discussão foi definido que seria replicado apenas o estudo baseado em entrevistas semi-estruturadas para garantir o entendimento dos termos pelos entrevistados e a adequação com as definições do artigo base. Após a discussão, os entrevistadores acessaram o apêndice online deste para obter o guia de entrevista utilizado. Como forma de manter a consistência, foi utilizado o mesmo guia de entrevistas, assim como as mesmas definições utilizadas pelos autores. 

No início foi montado um arquivo com a definição de cada um dos termos envolvidos em língua portuguesa para ser utilizado como consulta caso haja alguma dúvida de definição de tópicos entre os entrevistadores. Após isso, o guia de entrevistas do estudo base também foi traduzido para português e utilizado durante as entrevistas. Um ponto importante a respeito da tradução é o fato de que alguns termos ainda foram mantidos na língua inglesa devido ao fato de serem conhecidos mundialmente nesta língua. Exemplos: Continuous Integration, Canary Releases e Health check.
O documento de consulta e a guia de entrevistas estão presentes no apêndice deste trabalho, ambos em sua versão traduzida***.

\section{Estrutura da Entrevista}

O guia de entrevistas se baseia na estratégia de evitar perguntas diretas (exemplo: “determinada prática está sendo utilizada?”). Esse modelo é essencial para garantir que a comparação entre participantes a respeito do uso ou não de práticas está sendo feita de maneira concisa, e não baseada nos conhecimentos prévios de cada participante.

A entrevista é dividida em 5 sessões: Processo de entrega no geral, Papéis/Responsabilidades, 
Garantia da Qualidade (QA), Gerenciamentos de Problemas e Avaliação de Entrega. Todas as sessões começam com uma questão aberta. As entrevistas seguiram os tópicos abordados em cada uma delas, mas sem ordem específica, respeitando o desenrolar da conversa.

Todas as entrevistas ocorreram de forma online, através do Google Meet, e aconteceram entre os meses de Setembro e Outubro de 2020 em português brasileiro. Cada uma delas foi gravada pela plataforma para futura análise e, quando necessário, foram transcritas para o uso em citações neste trabalho. Foi deixado claro em cada uma das conversas a respeito da gravação e que estas seriam utilizadas apenas pelos entrevistadores, respeitando assim o anonimato de informações pessoais como nome ou empresa da qual o profissional se referia.

\section{Análise dos Dados}

A analise foi feita... 


\section{Participantes}

No total foram entrevistados 11 desenvolvedores (P1 a P11) - 3 mulheres - que trabalhavam em empresas de Recife de diversos domínios e tamanhos. 

<Adicionar dados demograficos>
