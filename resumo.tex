\resumo

[WIP - Retirado da proposta TG]

As práticas de integração, entrega e implantação continuas (\emph{Continuous Integration, Continuous Delivery} e \emph{Continuous Deployment}) já estão presentes no cotidiano de grandes empresas de todo o mundo. E isto é devido, entre outras coisas, aos comprovados benefícios que a utilização destas técnicas trazem para a equipe e para o produto em desenvolvimento. Mesmo assim, poucos são os estudos que pesquisam a respeito do estado da arte destas práticas nas empresas, principalmente no contexto da cidade de Recife. 

Há também um grande equívoco em relação a nomenclatura das práticas e a real utilização das mesmas em vários aspectos no processo de desenvolvimento software. A utilização de termos e softwares voltados para práticas que não são seguidas de fato por muitos projetos que os utilizam levam a um ambiente não saudável de desenvolvimento. 

Neste contexto, o presente trabalho objetiva a realização de uma pesquisa qualitativa para entender como as práticas de integração, entrega e implantação continuas são implantadas no contexto recifense, assim como avaliar o entendimento dos termos utilizados com os desenvolvedores locais.

% Palavras-chave do resumo em Português
\begin{keywords}
    DIGITE AS PALAVRAS-CHAVE AQUI
\end{keywords}
