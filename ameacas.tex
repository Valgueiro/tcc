\section{Ameaças a validade}

Como em qualquer estudo, este também apresenta algumas limitações. Uma delas diz respeito ao número de desenvolvedores que concordaram participar do estudo. Para diminuir os efeitos disso, foi escolhido uma amostra mais diversa em relação a tamanho da corporação que trabalha.

É importante salientar também que os códigos gerados durante o processo de codificação das entrevistas são passíveis de enviesamento visto que este trabalho replica parte de um estudo anterior. O viés ocorre porque os autores já tinham em mente que assuntos estavam sendo procurados na fala durante o levantamento de códigos. 

Outra ameaça que deve ser levada em conta é o processo de \emph{coding} realizado. Ele foi feito baseando-se no áudio das entrevistas, e não nos textos transcritos, como geralmente é feito \cite{groundedTheory}. Isso pode tornar os códigos enviesados ou até mesmo significar a falta de códigos importantes que poderiam ter sido levantados através da análise de transcrições. É válido citar que há algumas vertentes que defendem a codificação através do áudio \cite{listenCode}. Entre os benefícios de realizar o processo de codificação por essa técnica estão a preservação da entonação e intenção do usuário, que são aspectos difíceis de capturar por meio de transcrições. Contudo ainda devem ser feitos trabalhos quantitativos na área para validação e melhor definição do processo 
