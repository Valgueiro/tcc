% Resumo em Inglês

\begin{abstract}
The practices of Continuous Integration [CI], Continuous Delivery and Continuous Deployment [CD] are already present in the daily lives of large companies around the world. This is due, among other things, to the proven benefits that the use of these techniques bring to the team and to the product under development. Even so, there are few studies that investigate the state of the art of these practices in most companies, especially in the context of Recife. With this, the present work aims to understand how the techniques of continuous integration, delivery and  deployment were imported for Recife companies, as well as to identify underlying principles and practices that govern the adoption of these techniques. For this, a qualitative research was carried out through interviews with 11 software developers from technology companies based in the city of Recife. It was discovered that the sample does not follow the Stairway to Heaven, a theory defined by the base article of this work and suggests that adoption should follow a specific sequence of practices. Nevertheless, the majority interviewed integrates the code of a new functionality to the main branch only in the end of the Sprint, not daily, which is not consistent with the most common definitions of CI. Furthermore, the practice of A/B Tests was not found in any of the interviewed teams, due to the low number of active users or because it does not apply to the application context.


\end{abstract}

\keywords{Continuous Integration, Continuous Deployment, Collaborative Development, Software Engineering}

    