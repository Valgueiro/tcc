
\section{Conclusão}

Com os resultados encontrados, é possível perceber que, apesar de algumas poucas diferenças a amostra deste trabalho se comportou de forma condizente àquela encontrada no artigo base \cite{empiricalStudy2016}. Desde o ranking de utilização de cada prática até as barreiras enfrentadas pelos desenvolvedores, houve uma congruência razoável entre os dois resultados. Esta congruência, principalmente dentro do grupo das práticas mais utilizadas, serve como indicador de que elas acharam seu lugar na indústria e, assim como a teoria propunha, funcionam de forma eficiente.

Sobre o processo de integração contínua, pode-se inferir que as técnicas ligadas ao \emph{merge} contínuo estão pouco presentes, apesar de \emph{Developer Awareness} ser a prática mais utilizada da amostra. A maioria entrevistada integra o código na \emph{branch} principal apenas no final da \emph{sprint}. Este comportamento pode trazer vários problemas de integração de código quando atualizado concorrentemente, causa principal para a criação da prática de \emph{Trunk Based Development} \cite{devAndDeploymentFB}. É válido um trabalho mais quantitativo na região que busque razões específicas para a não utilização desta, aliado a uma divulgação maior dos benefícios que essa técnica pode trazer para times de desenvolvimento. 

É possível perceber ainda que todas as práticas ligadas à implantação contínua estão muito presentes na amostra. Mesmo com a prática de \emph{Developer on Call} sendo mais utilizada de forma implícita do que de fato definida, e com algumas empresas interpretando-a como uma má prática, ela está em 4º lugar no ranking de utilização montado e é a de menor colocação do grupo. Isso demonstra principalmente que, na amostra, há um trabalho bem evoluído de otimização e automação do processo de entrega, além de utilização de ferramentas para garantir que a aplicação está rodando corretamente em produção.

Por fim, foi possível perceber que os entrevistados utilizam muito pouco as práticas de entregas parciais. Foi possível identificar técnicas desconhecidas por um grupo razoável de participantes -- \emph{Dark Launches} -- e até técnicas que não foram utilizadas por nenhum deles -- Testes A/B. É necessário um trabalho maior de disseminação de como realizar e das vantagens que essas técnicas podem trazer para produtos produzidos em Recife.

Todos esses pontos demonstram uma certa congruência da amostra de Recife com a amostra européia e norte-americana do artigo base. Assim, podemos afirmar que a amostra está no mesmo patamar no sentido de utilização de outras partes do mundo, mas é necessário uma pesquisa quantitativa para poder generalizar esse achado para a capital pernambucana. 

\subsection{Trabalhos Futuros}

Como trabalhos futuros, pode ser feito o mesmo estilo de entrevistas com um grupo maior de pessoas. A reaplicação poderia levantar novas barreiras e até identificar lacunas que não puderam ser vistas com a amostra deste trabalho. Outro ponto que também seria de grande valia é a pesquisa sobre o mesmo tema, mas de forma quantitativa, com o objetivo de entender melhor que práticas estão sendo utilizadas na indústria de Recife com um grupo maior de entrevistados, permitindo assim a generalização dos achados para a capital. 
