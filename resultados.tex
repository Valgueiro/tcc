\section{Resultados}

Esta seção responde às perguntas de pesquisa apresentadas na Seção 2, de acordo com a metodologia descrita na seção anterior.

\subsection{Estudo de Predomínio das Práticas}

Com o intuito de responder a pergunta de pesquisa PP1 -- \emph{Quais as práticas de CI/CD são utilizadas pelas empresas em Recife?} foi montada a Tabela \ref{tabela_t3}, que apresenta a utilização de cada uma das práticas investigadas para cada um dos participantes, de acordo com a escala de uso apresentada na seção anterior. Assim, o verde mais escuro significa que o participante utiliza totalmente, enquanto o verde claro significa utiliza parcialmente e, por fim, o branco denota a não utilização.

\begin{table}[ht]
\begin{center}
\includegraphics[width=\linewidth]{tabelaT3.png}
\end{center}
\caption[Nível de utilização das práticas, com as colunas em ordem decrescente de uso]{
    Nível de utilização de cada uma das práticas, com as colunas ordenadas em ordem decrescente de uso.
}\label{tabela_t3}
\end{table}

\begin{figure}[ht]
\begin{center}
\includegraphics[width=0.9\linewidth]{tabela1-artigo-base.png}
\end{center}
\caption[Tabela 1 do estudo original]{
    Utilização das práticas pelos participantes do estudo original \cite{empiricalStudy2016}.
}\label{tabela_1_artigo_base}
\end{figure}

Com a Tabela \ref{tabela_t3} é possível perceber que \emph{Developer awareness [AWA]} foi a prática mais encontrada em toda a amostra, sendo totalmente utilizada pela grande maioria dos entrevistados; apenas um deles utiliza parcialmente. Logo após, pode-se encontrar as práticas de \emph{Health Check} [HC] e Pipeline de Implantação [PIP]  na segunda e na terceira colocações, respectivamente. No geral, também pode-se inferir que as técnicas de \emph{Partial Rollouts} são ainda muito pouco utilizadas, com as 3 práticas do grupo entre as quatro últimas colocadas. Vale a pena citar que nenhum dos entrevistados utiliza, mesmo que precariamente, a técnica de Testes A/B [AB].

Quando comparamos os resultados da Tabela \ref{tabela_t3} com os resultados do estudo original (Figura \ref{tabela_1_artigo_base}), podemos perceber que há diferenças de posição entre práticas, mas que não há mudanças exorbitantes. Com a ajuda da Tabela \ref{diferenca_entre_posicoes_fig}, é possível perceber que há uma diferença de no máximo 2 posições na ordem de predomínio, ao comparar os resultados obtidos neste trabalho com os do estudo original. Por exemplo, é possível perceber que as práticas de AWA, TBD, DAR e FT tiveram mudanças de 2 posições, enquanto HC, PIP, CAN e AB tiveram apenas diferença de 1 unidade de posição. Apenas \emph{developer on call} permaneceu no mesmo local dentro das duas amostras. Por fim, é interessante perceber que o conjunto das três primeiras práticas é o mesmo, mas em posições trocadas nos dois estudos.

\begin{table}[ht]
\begin{center}
\includegraphics[width=0.9\linewidth]{diferenca_entre_posicoes.png}
\end{center}
\caption[Diferença entre a ordem de predomínio das práticas]{
    Diferença entre o estudo original e este trabalho a respeito da ordem de predomínio das práticas.
}\label{diferenca_entre_posicoes_fig}
\end{table}

\subsection{Stairway to heaven}

Com o objetivo de responder a pergunta de pesquisa PP2 -- \emph{O cenário de CI/CD nas empresas em Recife segue o ``stairway to heaven'' proposto no artigo?} -- foi produzida a Tabela \ref{tabela_t2}. Esta contém a visualização de utilização das práticas, com o mesmo conteúdo da Tabela \ref{tabela_t3} e utilizando a mesma escala de grau de utilização, mas com uma outra perspectiva de visualização dos resultados que consiste na apresentação com as colunas ordenadas pela escada definida pela sequência de práticas do \emph{Stairway to Heaven} apresentada na Figura \ref{stairway}.

\begin{table}[ht]
\begin{center}
\includegraphics[width=0.95\linewidth]{tabelaT2.png}
\end{center}
\caption[Nível de utilização das práticas, com as colunas na ordem do \emph{Stairway to Heaven}]{
    Nível de utilização de cada uma das práticas, com as colunas ordenadas na ordem do \emph{Stairway to Heaven}.
}\label{tabela_t2}
\end{table}

Com a Tabela \ref{tabela_t2} é possível perceber que a amostra deste estudo, assim como a amostra do estudo original, não segue a evolução proposta pelos seus autores. Isso fica claro quando percebe-se que não há, em nenhum dos entrevistados, uma relação clara entre a coloração da coluna com a sua anterior. É possível notar também que há uma lacuna nas duas primeiras práticas, seguido de uma grande utilização da terceira, reforçando que a \emph{Stairway to Heaven} não foi identificada na amostra. O mais próximo dos participantes a atingir a escada definida é P5, mas ainda há nele a falta do pilar TBD do primeiro degrau da escada. É interessante destacar que na amostra do próprio estudo original também não foi possível identificar a escada de evolução.

\subsection{Análise das Práticas}

Nesta seção será abordado, para cada uma das práticas apresentadas na Tabela \ref{tabela_t2}, as principais características encontradas na amostra. Esta tem como objetivo responder a pergunta PP3: \emph{Quais são os princípios e práticas subjacentes que governam a adoção de CI/CD na indústria?}.

Para tanto, os códigos e super categorias criados durante a análise das entrevistas foram revisitados em busca de princípios que governam a adoção ou não de cada prática. Onde aplicado, ao longo desta seção, nós também destacamos como uma data prática se apresentou no estudo original, de modo a destacar as discrepâncias e congruências entre os contextos analisados por cada estudo.

\subsubsection{Integração Contínua}

Com a Tabela \ref{tabela_t3} é possível perceber que, apesar da grande disseminação a respeito de informações relacionadas à infraestrutura e manutenção de software -- ligadas à prática de \emph{Developer Awareness} \cite{awa} -- estar em primeiro lugar, as outras práticas que compõem o conjunto de integração contínua ainda não estão disseminadas na nossa amostra.

\paragraph{Trunk Based Development [TBD]}
Na amostra é possível perceber que a técnica de \emph{Trunk Based Development} [TBD] \cite{devAndDeploymentFB} não é tão amplamente adotada, visto que apenas 4 entrevistados utilizam ao menos parcialmente. Destes, 2 comentam que entregam novas versões aos clientes baseado em novas funcionalidades, e não em \emph{sprints}. Já entre os que não utilizam, 5 integram o código apenas no final da \emph{sprint}. Deste grupo, 2 utilizam a metodologia \emph{Git Flow} \cite{gitFlow}, que define uma maneira de manusear várias branches ao mesmo tempo de modo que os desenvolvedores deparem-se com o mínimo de conflitos possível e que software seja entregue em versões bem definidas.

\paragraph{Feature Toggles [FT]}

No estudo foi possível perceber que, na amostra, a prática de \emph{Feature Toggles} [FT] \cite{featureToggles} é raramente utilizada, assim como no estudo original. Esta técnica foi encontrada apenas na equipe do participante P5, que a utiliza para esconder funcionalidades enquanto testes manuais ainda estão sendo feitos. Geralmente esta técnica é utilizada para auxiliar a integração de códigos ainda não finalizados quando a equipe utiliza técnicas como o \emph{trunk based development}, mas este não é o caso de P5. É interessante notar que este participante também é um dos poucos que utiliza a técnica de \emph{Canary Releases} [CAN].

Entre o grupo dos que não utilizavam, 7 só enviam código novo para produção quando a funcionalidade está concluída. Deste grupo, 2 comentam que fazem uso de conceito de épicos, onde uma história de usuário se prolonga para além de apenas uma sprint. É também interessante notar que P3 está em vias de utilizar esta técnica para reduzir conflitos de \emph{merge} e \emph{rebase} devido ao grande número de desenvolvedores em seu time, como podemos ver na citação:


\begin{quote}
    ``O meu time tem 23 [pessoas]. [...] a gente tá trabalhando em funcionalidades muito distintas, então às vezes acontece de termos um paralelismo de branches muito grande. [...] é muito complicado `mergear' e fazer \emph{rebase} de tudo. Realmente dá muitos conflito'' --- P3 -- Web -- CORP
\end{quote}


\paragraph{Developer Awareness [AWA]}

Na amostra é possível identificar que a prática de \emph{Developer Awareness} [AWA] é amplamente adotada nas companhias, assim como na amostra do estudo original. Em 6 entrevistas foi possível perceber que o time de desenvolvimento era o mesmo responsável pela entrega e manutenção da aplicação. Em outras 2, há um time específico de \emph{DevOps}, mas não havia grandes silos entre este e a equipe de desenvolvimento. Um caso interessante foi o de P11, que, apesar de um conhecimento espalhado dentro da equipe, há receio e insegurança por parte de alguns a respeito de questões de infraestrutura no geral.


\subsubsection{Implantação Contínua}

É possível inferir, baseado principalmente na Tabela \ref{tabela_t3}, que as técnicas ligadas ao processo de implantação contínua estão presentes na maioria das equipes da amostra: as 3 estão nas quatro primeiras colocações. 

\paragraph{Health Checks [HC]}

Sobre a prática de \emph{Health Checks} [HC] \cite{devopsBook}, é possível perceber que, apesar de não ser o mais adotado -- como foi no estudo original (ver Figura \ref{tabela_1_artigo_base}), se destaca na nossa amostra, em relação à maioria das prática, presente na segunda colocação. Na amostra, 7 entrevistados adotam pelo menos uma forma rudimentar de verificação e alertas, e 2 entre eles fazem verificações não muito complexas. Um ponto interessante que surgiu foi o fato de P4 achar desnecessário o uso dessa técnica em função do projeto se encontrar, à época da entrevista, em fase de prototipação.

\paragraph{Developer on Call [DOC]}

Na amostra é possível perceber que a técnica de \emph{Developer on Call} [DOC] \cite{devAndDeploymentFB} é mais adotada de forma implícita do que de fato definida -- 3 entrevistados estão em times que funcionam desta forma. Contudo, 5 entrevistados não utilizam esta prática: alguns comentaram que a confiança nos testes automáticos faz com que não utilizem a prática, enquanto outro comentou que a prática é inclusive mal vista pela empresa.

Uma dicotomia interessante foi encontrada entre os resultados da amostra deste trabalho e o do estudo original. Este último comenta que a prática já está sendo largamente aceita nas organizações atualmente, e inclusive um dos entrevistados comenta que essa responsabilidade de ficar até mais tarde para resolver problemas leva os desenvolvedores a escreverem e testarem seus códigos mais veementemente. Tal argumentação tem como base a seguinte citação retirada do artigo e traduzida:

\begin{quote}
    ``Se você não tem testes suficientes e faz deploy de um código ruim isso vai se voltar contra você pois você estará de plantão e terá que dar suporte a isto.'' --- P14 (do estudo original) -- Web -- CORP
\end{quote}


Contudo, com a seguinte citação de P2, é possível perceber que há um sentimento contrário: confia-se no processo de qualidade e, por isso, não necessitam de plantão.


\begin{quote}
    ``Temos o ciclo de QA, se encontrar alguma coisa a gente vê […] não precisa isso de plantão não.'' --- P2 -- Web -- CORP
\end{quote}

\paragraph{Pipeline de Implantação [PIP]}

Em relação à prática de Pipeline de Implantação [PIP] \cite{devopsBook} é possível inferir que ela é amplamente adotada pela amostra, visto que apenas 2 entrevistados obtiveram nota 0 (não utiliza). No estudo original é possível perceber um resultado semelhante a este. Uma grande parte dos entrevistados segue o mesmo padrão para qualquer tamanho da mudança. Outros 2 tinham alguns processos automatizados, mas a \emph{pipeline} era diferente dependendo do tamanho da mudança.

É interessante perceber que alguns times ainda demonstram falta de automação dos processos da \emph{pipeline}. Isto acontece em decorrência da complexidade da automação, devido às tecnologias e ferramentas utilizadas, ou da falta de prioridade do time para tal.
 
\subsubsection{Entregas Parciais}

Na amostra foi possível perceber que todas as técnicas relacionadas a Entregas Parciais são muito pouco utilizadas: todas estão entre as 4 últimas colocações.

\paragraph{Canary Releases [CAN]}

A técnica de \emph{Canary Releases} [CAN] \cite{continuousDeliveryBook} apareceu nos contextos de jogos e sistemas embarcados. Na equipe do entrevistado P5, que trabalha no domínio de jogos eletrônicos, os principais jogadores -- conhecidos como ``baleias'' -- são escolhidos para participar de um \emph{early access} de novas funcionalidades.  Esses testes levam em torno de 1 semana. 

Já na equipe de P4, que trabalha com sistemas embarcados, a prática é necessária devido ao contexto de atuação e às tecnologias utilizadas. Como o sistema que está em fase de prototipação servirá para o contexto médico, vários testes de campo deveriam ser feitos para garantir que todas as funcionalidades estivessem de acordo com o esperado. Para os testes, o cliente que contratou a empresa de P4 escolhia a quantidade de pessoas e local que serviria como validação de funcionalidades.

Entre os entrevistados que não utilizam a prática de \emph{Canary Releases}, 3 têm um ambiente de homologação para testes e validação de requisitos, mas este utiliza dados diferentes dos de produção. Outros 3 comentam que as features são sempre entregues para todos os usuários ao mesmo tempo.


\paragraph{Dark Launches [DAR]}

Do grupo das práticas associadas a Entregas Parciais, \emph{Dark Launches} [DAR] \cite{devAndDeploymentFB} foi a segunda mais utilizada, mas a menos conhecida entre os entrevistados. Isto se confirma com o fato de que 6 entrevistados dizem nunca ter utilizado, e 3 destes afirmam explicitamente que não conhecem a técnica. Importante notar que no estudo original \cite{empiricalStudy2016} esta é a prática menos utilizada.

\emph{Dark Launches} é utilizado totalmente por P5 no contexto de jogos para testes manuais, e apenas parcialmente no contexto de WEB por P9 para validações de alguns cenários que dependiam de dados de produção.


\begin{quote}
    ``A gente implementa a funcionalidade mas condiciona a não aparecer para o usuário até que a gente queira.'' --- P5 -- Jogos -- Startup
\end{quote}

\paragraph{Testes A/B [AB]}

A utilização da prática de Testes A/B [AB] \cite{testsAB} não foi identificada entre os participantes. Entre as principais causas levantadas para a não utilização, 4 entrevistados dizem que não utilizam pela baixa quantidade de usuários ativos no sistema. Outros 2 consideram que a técnica não se aplica ao contexto da aplicação. É interessante notar que esses dois motivos estão presentes no estudo original \cite{empiricalStudy2016}, contudo, ao contrário deste, ninguém na amostra comentou sobre problemas na arquitetura como causa para não utilização.


\begin{quote}
    ``O sistema da gente -- apesar de lidar com uma massa de dados muito grande -- não têm tantos usuários, então não faz muito sentido [utilizar testes A/B]....'' --- P2 -- WEB -- Corp
\end{quote}

Outro motivo importante foi levantado por P8, que comenta que aparentemente não existe na empresa o interesse em investir nessa prática. P3 comenta ainda que o time está mais focado em entregar novas funcionalidades, pois a demanda por parte do cliente é muito grande.

\subsection{Descobertas adicionais}

Esta subseção abordará sobre os dois tópicos marginais que foram identificados durante a análise dos dados obtidos pelas entrevistas: \emph{Code Review} e testes automáticos.

\subsubsection{Code Review}

A prática de \emph{Code Review} \cite{codeReview} é bem vista pela maioria dos entrevistados, considerado uma técnica importante e essencial para o controle de qualidade de código. As razões para o uso são diversas, desde de impor boas práticas -- por P7 -- até o compartilhamento e nivelamento do conhecimento -- por P11.

\begin{quote}
    ``... a partir do momento que o sistema vai crescendo, o próprio desenvolvedor não tem a noção de que aquela sua mudança não é a melhor forma de fazer e que pode quebrar outras partes do sistema...'' --- P2 -- WEB -- Corp
\end{quote}

Ainda na amostra, 3 entrevistados acreditam que a prática funciona principalmente para troca de conhecimento. Sobre isso, a entrevistada P10 comenta que agrega muito valor para ela como novata no time, mas acha que adiciona um tempo desnecessário na entrega de funcionalidades por pessoas mais experientes, visto que -- para ela -- a técnica não faria sentido neste caso.

Outro ponto importante foi a distinção entre dois relatos a respeito do engajamento do time com a prática. Enquanto P3 comenta que a equipe dela é bem aberta ao debate e está engajada com o processo, P7 fala que o processo funciona ``mais ou menos'', dependendo do humor dos revisores.

\subsubsection{Testes Automáticos}

Testes automáticos são, no geral, bem vistos e extremamente recomendados pela maioria dos entrevistados como forma de prevenir erros em tempo de execução. Contudo, o participante P2 comenta que é contrário ao fato da garantia de qualidade depender exclusivamente desses testes.

\begin{quote}
    ``... automação [de testes] não resolve todos os problemas [relacionados a garantia de qualidade], ele vai identificar muita coisa, mas tem várias outras que precisamos do olhar de um testador…'' --- P2 -- WEB -- Corp
\end{quote}

Na amostra, 5 entrevistados acreditam que o aumento da cobertura de testes automáticos pode diminuir a quantidade de bugs em produção. Dentro desse grupo, o participante P9 comenta que isto poderia, no entanto, diminuir a velocidade de entregas do time. P5 adicionou ainda que sente que a \emph{sprint} é muito curta e não consegue tempo dentro destas para adicionar testes automáticos.