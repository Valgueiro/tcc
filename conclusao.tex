
\section{Conclusão}

Com a realização deste estudo, percebeu-se dentro da amostra a presença de pelo menos alguma das técnicas de CI/CD presentes no cotidiano das empresas Recifenses. Com a análise da entrevista qualitativa, foi possível entender um pouco melhor como tais técnicas foram importadas e também algumas das razões do porque foram adotadas. Ainda que a amostra não siga fielmente o \emph{Stairway to Heaven} proposto no estudo original \cite{empiricalStudy2016}, observou-se certa congruência no que diz respeito ao resultado de ambos os estudos.

Além disso, percebeu-se que testes A/B não eram utilizados, além de que \emph{trunk based development} se mostrou pouco utilizado, visto que a maioria entrevistada relatou que a integração do código de uma nova funcionalidade para a \emph{branch} principal de desenvolvimento era feita apenas no final da \emph{Sprint}.