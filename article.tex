%%
%% This is file `sample-acmlarge.tex',
%% generated with the docstrip utility.
%%
%% The original source files were:
%%
%% samples.dtx  (with options: `acmlarge')
%% 
%% IMPORTANT NOTICE:
%% 
%% For the copyright see the source file.
%% 
%% Any modified versions of this file must be renamed
%% with new filenames distinct from sample-acmlarge.tex.
%% 
%% For distribution of the original source see the terms
%% for copying and modification in the file samples.dtx.
%% 
%% This generated file may be distributed as long as the
%% original source files, as listed above, are part of the
%% same distribution. (The sources need not necessarily be
%% in the same archive or directory.)
%%
%% The first command in your LaTeX source must be the \documentclass command.
\documentclass[sigconf]{acmart}

%%
%% \BibTeX command to typeset BibTeX logo in the docs
\AtBeginDocument{%
  \providecommand\BibTeX{{%
    \normalfont B\kern-0.5em{\scshape i\kern-0.25em b}\kern-0.8em\TeX}}}

%% Rights management information.  This information is sent to you
%% when you complete the rights form.  These commands have SAMPLE
%% values in them; it is your responsibility as an author to replace
%% the commands and values with those provided to you when you
%% complete the rights form.
%% TODO review later

% \setcopyright{acmcopyright}
% \copyrightyear{2018}
% \acmYear{2018}
% \acmDOI{10.1145/1122445.1122456}


%%
%% These commands are for a JOURNAL article.
%% TODO review later
% \acmJournal{POMACS}
% \acmVolume{37}
% \acmNumber{4}
% \acmArticle{111}
% \acmMonth{8}

%%
%% Submission ID.
%% Use this when submitting an article to a sponsored event. You'll
%% receive a unique submission ID from the organizers
%% of the event, and this ID should be used as the parameter to this command.
%%\acmSubmissionID{123-A56-BU3}

%%
%% The majority of ACM publications use numbered citations and
%% references.  The command \citestyle{authoryear} switches to the
%% "author year" style.
%%
%% If you are preparing content for an event
%% sponsored by ACM SIGGRAPH, you must use the "author year" style of
%% citations and references.
%% Uncommenting
%% the next command will enable that style.
%%\citestyle{acmauthoryear}
\graphicspath{ {./images/} }
\DeclareGraphicsExtensions{.png}
% \usepackage{cite}
% \usepackage[pdftex]{hyperref}
% \usepackage{xcolor}
% \hypersetup{
%     colorlinks,
%     linkcolor={red!50!black},
%     citecolor={blue!50!black},
%     urlcolor={blue!80!black}
% }
\usepackage[autostyle]{csquotes}

%encoding
%--------------------------------------
\usepackage[T1]{fontenc}
\usepackage[utf8]{inputenc}
%--------------------------------------

%Portuguese-specific commands
%--------------------------------------
\usepackage[portuguese]{babel}
%--------------------------------------

%Hyphenation rules
%--------------------------------------
\usepackage{hyphenat}
\hyphenation{mate-mática recu-perar}
%--------------------------------------
 


\usepackage{titlesec}

\setcounter{secnumdepth}{4}

\titleformat{\subsubsection}
{\normalfont\normalsize\bfseries}{\thesubsubsection}{1.2em}{}
\titlespacing*{\subsubsection}
{0pt}{3.25ex plus 1ex minus .2ex}{1.5ex plus .2ex}

\titleformat{\paragraph}
{\normalfont\normalsize\bfseries}{\theparagraph}{1em}{}
\titlespacing*{\paragraph}
{0pt}{3.25ex plus 1ex minus .2ex}{1.5ex plus .2ex}

%%
%% end of the preamble, start of the body of the document source.
\begin{document}


%%
%% The "title" command has an optional parameter,
%% allowing the author to define a "short title" to be used in page headers.
\title{Pesquisa sobre práticas de Integração e Deployment contínuos em Recife}

%%
%% The "author" command and its associated commands are used to define
%% the authors and their affiliations.
%% Of note is the shared affiliation of the first two authors, and the
%% "authornote" and "authornotemark" commands
%% used to denote shared contribution to the research.
% \author{Mateus Valgueiro Teixeira}
% \email{mvt@cin.ufpe.br}
% \orcid{1234-5678-9012}
% \affiliation{%
%   \institution{Centro de Informatica}
%   \streetaddress{Av. Jorn. Aníbal Fernandes}
%   \city{Recife}
%   \state{Pernambuco}
%   \country{Brazil}
% }

% \author{Heitor Sammuel Carvalho Souza}
% \email{hssc@cin.ufpe.br}
% \orcid{1234-5678-9012}
% \affiliation{%
%   \institution{Centro de Informatica}
%   \streetaddress{Av. Jorn. Aníbal Fernandes}
%   \city{Recife}
%   \state{Pernambuco}
%   \country{Brazil}
% }

% \author{Paulo Borba}
% \email{phmb@cin.ufpe.br}
% \orcid{1234-5678-9012}
% \affiliation{%
%   \institution{Centro de Informatica}
%   \streetaddress{Av. Jorn. Aníbal Fernandes}
%   \city{Recife}
%   \state{Pernambuco}
%   \country{Brazil}
% }

% \author{Klissiomara Lopes Dias}
% \email{kld2@cin.ufpe.br}
% \orcid{1234-5678-9012}
% \affiliation{%
%   \institution{Centro de Informatica}
%   \streetaddress{Av. Jorn. Aníbal Fernandes}
%   \city{Recife}
%   \state{Pernambuco}
%   \country{Brazil}
% }

%%
%% By default, the full list of authors will be used in the page
%% headers. Often, this list is too long, and will overlap
%% other information printed in the page headers. This command allows
%% the author to define a more concise list
%% of authors' names for this purpose.
% \renewcommand{\shortauthors}{Mateus Valgueiro et al.}

%%
%% The abstract is a short summary of the work to be presented in the
%% article.
\resumo

As práticas de \emph{Continuous Integration} [CI], \emph{Continuous Delivery} e \emph{Continuous Deployment} [CD] já estão presentes no cotidiano de grandes empresas de todo o mundo. E isto é devido, entre outras coisas, aos comprovados benefícios que a utilização destas técnicas trazem para a equipe e para o produto em desenvolvimento. Mesmo assim, poucos são os estudos que investigam o estado da arte destas práticas na maioria das empresas, principalmente no contexto da cidade de Recife.  Com isso, o presente trabalho objetiva entender como as técnicas de integração, entrega e implantação contínuas foram importadas para as empresas recifenses, assim como identificar princípios e práticas adjacentes que governam a adoção destas técnicas. Para tanto, foi realizada uma pesquisa qualitativa por meio de entrevistas com 11 desenvolvedores de software de empresas de tecnologia sediadas na cidade do Recife. Foi descoberto que a amostra não segue o \emph{Stairway to Heaven}, teoria definida pelo artigo base deste trabalho e que sugere que a adoção deve seguir uma sequência específica de práticas. Não obstante, a maioria entrevistada integra o código de uma nova funcionalidade para a \emph{branch} principal apenas no final da \emph{Sprint}, não diariamente, o que não é consistente com as definições mais comuns de CI. Além disso, as práticas de entregas parciais são utilizadas apenas por uma pequena parcela de desenvolvedores.

% Palavras-chave do resumo em Português
\begin{keywords}
    Integração Contínua, Deployment Contínuo, Desenvolvimento Colaborativo, Engenharia de Software
\end{keywords}


%%
%% The code below is generated by the tool at http://dl.acm.org/ccs.cfm.
%% Please copy and paste the code instead of the example below.
%%
% \begin{CCSXML}
% <ccs2012>
%  <concept>
%   <concept_id>10010520.10010553.10010562</concept_id>
%   <concept_desc>Computer systems organization~Embedded systems</concept_desc>
%   <concept_significance>500</concept_significance>
%  </concept>
%  <concept>
%   <concept_id>10010520.10010575.10010755</concept_id>
%   <concept_desc>Computer systems organization~Redundancy</concept_desc>
%   <concept_significance>300</concept_significance>
%  </concept>
%  <concept>
%   <concept_id>10010520.10010553.10010554</concept_id>
%   <concept_desc>Computer systems organization~Robotics</concept_desc>
%   <concept_significance>100</concept_significance>
%  </concept>
%  <concept>
%   <concept_id>10003033.10003083.10003095</concept_id>
%   <concept_desc>Networks~Network reliability</concept_desc>
%   <concept_significance>100</concept_significance>
%  </concept>
% </ccs2012>
% \end{CCSXML}

% \ccsdesc[500]{Computer systems organization~Embedded systems}
% \ccsdesc[300]{Computer systems organization~Redundancy}
% \ccsdesc{Computer systems organization~Robotics}
% \ccsdesc[100]{Networks~Network reliability}

%%
%% Keywords. The author(s) should pick words that accurately describe
%% the work being presented. Separate the keywords with commas.


%%
%% This command processes the author and affiliation and title
%% information and builds the first part of the formatted document.
\maketitle


\chapter{Introdução}

As práticas de integração, entrega e implantação contínuos \cite{fowlerCI, fowlerCD} (\emph{Continuous Integration} [CI], \emph{Continuous Delivery} e \emph{Continuous Deployment} [CD]) já são muito difundidas e utilizadas por empresas de tecnologia em todo o mundo. Segundo a pesquisa da empresa \emph{digital.ia}\cite{stateAgileReport2020}, 55\% dos participantes reportaram que sua organização utiliza a técnica de integração contínua. Também nesta mesma pesquisa foi encontrado que 41\% utilizam a técnica de entrega contínua e 36\%, implantação.

Estes números elevados são causados principalmente pelos benefícios que a utilização destas técnicas trazem para a equipe e para o produto em desenvolvimento. Estudos comprovam que os desenvolvedores envolvidos se sentem mais produtivos quando usam as práticas de CI/CD \cite{hilton2016} e o seu uso traz maior qualidade ao software que está sendo produzido \cite{savor2015}. 

É possível, no entanto, perceber uma grande falta de estudos a respeito de como essas práticas foram importadas para as empresas de todo o mundo \cite{empiricalStudy2016}, incluindo Recife -- as pesquisas são geralmente voltadas apenas para as corporações globais como Facebook \cite{savor2015} e Google \cite{googleCi}. A capital pernambucana é um dos grandes pólos tecnológicos do Brasil: somente no Porto Digital, localizado na cidade, há cerca de 330 empresas e 11 mil trabalhadores, com faturamento anual de R\$ 2,3 bilhões em 2019 \cite{portoDigital}.

Um estudo a respeito de como as técnicas de CI/CD migraram para a indústria de Recife serve como ponto de partida para pesquisas relacionadas ao levantamento de dores sentidas pelos desenvolvedores locais que inibem a utilização destas técnicas.

Neste contexto, este trabalho tem como objetivo entender, através de uma pesquisa qualitativa, quais as práticas e técnicas subjacentes de integração, entrega e \emph{deployment} contínuos adentraram nas empresas de Recife. Não obstante, o trabalho deseja descobrir que princípios e práticas subjacentes governam a adoção destas técnicas.

\section{Estrutura do Trabalho}
 O trabalho está divido em 6 capítulos, segmentados da seguinte forma:

 \begin{itemize}
     \item O Capítulo 2 contém a Fundamentação Teórica deste trabalho, abordando sobre as práticas de CI/CD e definindo termos que serão utilizados durante o trabalho, além de trabalhos relacionados
     \item O Capítulo 3 apresenta a motivação por trás deste trabalho
     \item O Capítulo 4 contém a metodologia utilizada para a montagem e execução das entrevistas, além de todos os passos da análise de dados
     \item No Capítulo 5 encontram-se os resultados obtidos com base nas entrevistas
     \item O Capítulo 6 contém a conclusão com uma análise final e perspectivas de trabalhos futuros.
 \end{itemize}


\chapter{Motivação}
Neste capítulo será abordada a motivação por trás da pesquisa produzida. Na primeira seção, será falado um pouco sobre a grande adoção de CI/CD pela indústria. Já na segunda seção, será abordada a falta de estudos voltados para o contexto Recifense. A terceira discorre sobre o artigo base que serviu de motivação para este trabalho. A última seção contém as perguntas de pesquisa que este trabalho tenta responder.

\section{A grande adoção de CI e CD}
As práticas de integração e \emph{deployment} contínuos (\emph{Continuous Integration} e \emph{Continuous Deployment}) já são muito difundidas e utilizadas por empresas de tecnologia em todo o mundo. Segundo a pesquisa realizada pela empresa \emph{digital.ai} \cite{stateAgileReport2020}, 55\% dos participantes reportaram que sua organização pratica a técnica de integração contínua. Também nesta mesma pesquisa foi encontrado que 36\% utilizam a técnica de \emph{deployment} contínuos. 

Estes números elevados são causados principalmente pelos benefícios que a utilização destas técnicas trazem para a equipe e para o produto em desenvolvimento. Ainda de acordo com \cite{stateAgileReport2020}, entre as razões para adoção de CI/CD, as principais são aceleração de entregas de software (71\%), e aumentar a produtividade (51\%) e a qualidade do software (42\%). 

Especificamente sobre integração contínua, a prática hoje em dia já é bastante estudada difundida na indústria. O estudo \cite{hilton2016} comenta que desenvolvedores envolvidos se sentem mais produtivos quando utilizam a prática e dão mais valor aos testes automáticos.  Já com relação a \emph{deployment} contínuo, o estudo \cite{savor2015} mostra que seu uso traz maior qualidade ao software que está sendo produzido. 

\section{A falta de estudos no contexto Recifense}

Ainda há, no entanto, uma grande falta de estudos a respeito de como essas práticas foram importadas para a maioria das empresas \cite{empiricalStudy2016}, inclusive as de Recife, um dos grandes polos tecnológicos do Brasil. Somente no Porto Digital, localizado na capital pernambucana, há cerca de 330 empresas e 11 mil trabalhadores, com faturamento anual de R\$ 2,3 bilhões em 2019 \cite{portoDigital}.

O presente trabalho procura entender sobre a utilização das práticas de CI/CD nas empresas de Recife para que esses dados sirvam como objeto de pesquisa para levantamento de possíveis dores sentidas pelos desenvolvedores locais que justifiquem a adoção ou não destas práticas. Um estudo a respeito de como as técnicas de integração e \emph{deployment} contínuo migraram para a indústria de Recife funciona ainda como um \emph{benchmark} de como as empresas estão se portando em relação às novidades presentes na literatura nos últimos anos, assim como levantar possíveis discrepâncias entre empresas situadas na área e as grandes corporações. 


\section{O artigo base}
Com o objetivo de entender um pouco mais sobre o estado da prática de CI/CD no contexto de Recife, nos baseamos no estudo \cite{empiricalStudy2016}, que busca entender como as práticas geralmente associadas a \emph{Continuous Deployment} acharam o seu caminho nas indústrias européia e norte-americana. Nesse estudo os autores utilizaram um método misto de estudo empírico baseado em um pré-estudo na literatura, entrevistas com 20 participantes e um \emph{survey} que recebeu 187 respostas. A ideia era questionar até que ponto o conhecimento na área estava dominado por peculiaridades de um pequeno grupo de grandes empresas, como Facebook e Google.

Os autores também definem a chamada \emph{stairway to heaven} (escada para o céu, em tradução livre), presente na \figref{stairway}. Ela tem como objetivo definir um caminho de evolução das empresas para um estágio de entregas sofisticado. A escada permeia práticas de integração contínua, \emph{deployment} contínuo e entregas parciais.

\begin{figure}[ht]
\begin{center}
\includegraphics[width=\textwidth]{stairway_to_heaven.png}
\end{center}
\caption[Stairway to Heaven]{
    A escada de evolução denominada \emph{Stairway to Heaven} proposta pelo artigo base.
    Fonte: Schermman et al \cite{empiricalStudy2016}
}\label{stairway}

\end{figure}

Através desta metodologia os autores descobriram que, no contexto estudado, problemas arquiteturais são geralmente uma das maiores barreiras para a adoção de CD. Não obstante, a técnica de \emph{Feature Toggles} \cite{featureToggles} como forma de realizar entregas parciais adiciona uma complexidade demasiada e não saudável ao código. Por fim, eles concluem que os desenvolvedores necessitam também de um protocolo baseado em princípios que estabeleçam quando deve-se utilizar técnicas de entregas parciais, por exemplo, que funcionalidades e métricas devem ser testadas por um teste A/B \cite{testsAB}.

Aprender mais sobre as dificuldades que outras empresas não globais sofrem no processo de adoção de práticas de CI/CD pode gerar mais estudos a respeito de como solucionar tais problemas, assim como mostrar possíveis oportunidades de melhoria e revisão dos processos utilizados. Um trabalho que utilize uma adaptação da metodologia aplicada a uma amostra de um outro contexto pode revelar ainda discrepâncias e semelhanças entre os dois ambientes de estudo. As discrepâncias levantariam pontos de questionamento, pesquisa e até possíveis melhorias aos ambientes envolvidos. Já as semelhanças podem assegurar que as práticas utilizadas já acharam o seu lugar na indústria e funcionam bem assim como a teoria propunha.

Assim, a grande adoção da indústria mundial das práticas de CI/CD, aliado a falta de trabalhos a respeito no contexto recifense, além da metodologia já validada proposta pelo artigo foram as principais motivações para o desenvolvimento deste trabalho

\section{Perguntas de pesquisa} 
Com o intuito de entender como as técnicas de integração, entrega e \emph{deployment} contínuos foram importadas para as empresas recifenses, as seguintes perguntas de pesquisa foram formadas:

\begin{enumerate}
\item Quais as práticas de CI/CD são utilizadas pelas empresas em Recife?
\item O cenário de CI/CD nas empresas em Recife segue o \emph{stairway to heaven} proposto no artigo?
\item Quais são os princípios e práticas subjacentes que governam a adoção de CI/CD na indústria?
\end{enumerate}

Para responder às perguntas acima foi decidido aplicar a pesquisa qualitativa do artigo \cite{empiricalStudy2016} com desenvolvedores recifenses, baseando-se também na mesma lista de definições de cada uma das práticas envolvidas no \emph{Stairway to Heaven}. A pesquisa qualitativa neste contexto foi considerada fundamental para garantir que os entrevistados entendessem claramente as perguntas e excluir possíveis entendimentos errados de termos em inglês, linguagem não nativa de todos os participantes. Vale salientar que não foi replicado a pesquisa quantitativa presente no artigo base devido à falta de tempo hábil para tal, mas esta pode servir como um trabalho futuro a este.

Além disso, as perguntas 1 e 3 são uma adaptação das perguntas 1 e 2 do artigo base \cite{empiricalStudy2016}, respectivamente, incluindo a técnica de \emph{Continuous Integration}. Não obstante, uma terceira pergunta foi adicionada (\emph{RQ2}), que foca na escada de evolução proposta pelo artigo. Ela tenta responder se a \emph{stairway to heaven} é seguida no contexto de Recife, visto que, no outro, os próprios autores já refutaram esta definição com os seus resultados.


\chapter{Metodologia}

Para o estudo, foi realizada uma pesquisa qualitativa através de entrevistas semi-estruturadas utilizando como base a entrevista desenvolvida pelo artigo base \cite{empiricalStudy2016}. A primeira seção abordará uma visão geral sobre os primeiros passos da pesquisa.  Já a segunda seção discorre sobre como as entrevistas foram conduzidas. Na terceira, encontra-se todo o processo utilizado para a análise dos dados obtidos na entrevista. Por último, a quarta seção mostra informações a respeito dos participantes da pesquisa.


\section{Pré-estudo}

Com o objetivo de entender melhor sobre como as práticas de CI/CD migraram para as empresas sediadas em Recife, Mateus Valgueiro (autor deste trabalho) e Heitor Samuel realizaram um pré-estudo. Primeiramente, os dois discutiram a respeito do artigo base: \emph{An empirical study on principles and practices of continuous delivery and deployment} \cite{empiricalStudy2016}. Na discussão foi definido que seria replicado apenas o estudo baseado em entrevistas semi-estruturadas para garantir o entendimento dos termos pelos entrevistados e a adequação com as definições do artigo base. 

Após a discussão, os entrevistadores acessaram o apêndice online deste \cite{empiricalStudyOnlineAppendix} para obter o guia de entrevista utilizado. Como forma de manter a consistência, foi utilizado o mesmo guia de entrevistas, assim como as mesmas definições utilizadas pelos autores. No início foi montado um arquivo com a definição de cada um dos termos envolvidos em língua portuguesa para ser utilizado como consulta caso haja alguma dúvida de definição de tópicos entre os entrevistadores.

Após isso, o guia de entrevistas do estudo base \cite{empiricalStudyOnlineAppendix} também foi traduzido para português e utilizado durante as entrevistas. Um ponto importante a respeito da tradução é o fato de que alguns termos ainda foram mantidos na língua inglesa devido ao fato de serem conhecidos mundialmente nesta língua. Por exemplo, Continuous Integration, Canary Releases e Health check.
O documento de consulta e a guia de entrevistas estão presentes no apêndice deste trabalho, ambos em sua versão traduzida.

\section{Estrutura da Entrevista}

O guia de entrevistas se baseia na estratégia de evitar perguntas diretas (exemplo: “determinada prática está sendo utilizada?”). Esse modelo é essencial para garantir que a comparação entre participantes a respeito do uso ou não de práticas está sendo feita de maneira concisa, e não baseada nos conhecimentos prévios de cada participante. Assim, através de perguntas sobre o processo utilizado pelo entrevistado é possível, com uma certa margem de erro associada, afirmar que práticas ele utiliza. A margem de erro surge do fato de o autor ter que refletir sobre as informações recebidas e as definições para inferir o uso ou não de certa prática.

A entrevista é dividida em 5 sessões: 

\begin{enumerate}
\item Processo de entrega no geral
\item Papéis/Responsabilidades
\item Garantia da Qualidade (Quality Assurance)
\item Gerenciamentos de Problemas
\item Avaliação de Entrega
\end{enumerate}

No guia, todas as sessões iniciam com uma questão aberta. As entrevistas seguiram os tópicos abordados em cada uma delas, mas sem ordem específica, respeitando o desenrolar da conversa. A primeira entrevista foi guiada pelos dois pesquisadores para assegurar que as próximas seriam feitas de forma semelhante pelos dois. Esta entrevista foi considerada como válida na análise de dados visto que não houve nenhuma mudança no questionário de entrevistas; o próprio já tinha sido validado no artigo utilizado de base para este trabalho. As outras 10 - totalizando 11 entrevistas feitas - foram guiadas por apenas um entrevistador: 5 conduzidas por Heitor e outras 5 conduzidas por Mateus.


Todas as entrevistas ocorreram de forma online, através do Google Meet, e aconteceram entre os meses de Setembro e Outubro de 2020 em português brasileiro. As entrevistas levaram entre 30 e 50 minutos, duração bem parecida com os tempos obtidos no artigo base (35 a 60 minutos). Cada uma delas foi gravada pela plataforma para futura análise e 4 delas foram transcritas para o uso em citações neste trabalho, escolhidas através da relevância da entrevista e da forma como certos termos e processos foram apresentados pelo entrevistado. Foi deixado claro em cada uma das conversas a respeito da gravação e que estas seriam utilizadas apenas pelos entrevistadores, respeitando assim o anonimato de informações pessoais como nome ou empresa da qual o profissional se referia.

\section{Análise dos Dados}

\begin{figure}[ht]
\begin{center}
\includegraphics[scale=0.5]{metodologia_tcc.png}
\end{center}
\caption[Fluxograma da Metodologia]{
    Demonstra como funcionou o processo de coleta e análise de dados.
}\label{fluxograma_metodologia}
\end{figure}

    
A \figref{fluxograma_metodologia} apresenta uma visão geral de como funcionou o processo de coleta e análise de dados. A análise foi feita apenas pelo autor deste artigo, Mateus Valgueiro. O processo escolhido foi baseado nas fases de \emph{Coding} e \emph{Thematic Analysis} da metodologia \emph{Grounded Theory} \cite{groundedTheory}. No processo de Coding a ideia é levantar rótulos ou tags relevantes para o texto e, tradicionalmente, é feito baseado na transcrição das entrevistas. No entanto, neste trabalho o autor gerou códigos através da escuta das entrevistas. Então, como um exemplo, a seguinte citação:

\begin{quotation}[]{P5}
"Como somos uma equipe muito pequena, todos os desenvolvedores são meio que DEVOPS. Quando tem que tomar alguma decisão nós entramos em discussão e definimos por nós."
\end{quotation}

Gerou o código: \emph{"Todos os desenvolvedores são devops."} - P5\_15

É importante salientar que os códigos gerados sofreram um certo enviesamento visto que este trabalho é uma replicação de um estudo, então o autor tinha em mente que assuntos estavam sendo procuradas na fala durante o levantamento de códigos. 

Então, após levantados todos os códigos de uma entrevista, estes foram revisados para garantir semântica e sintaxe adequadas. Alguns códigos nessa fase foram eliminados por redundância, enquanto outros foram quebrados em múltiplos. Depois, eles foram agrupados, quando compatíveis, em cada uma das 9 práticas descritas pelo artigo base e foi escolhida uma nota entre 0 e 2, representando não utiliza, utiliza parcialmente e utiliza completamente baseado nas definições do artigo base traduzidas. Este processo foi então replicado para cada uma das 11 entrevistas.

Ao final do processo ainda haviam 3 ligações entre a prática de \emph{Dark Launch} e entrevistados que não tinham recebido nenhum código em comum. Para estes casos, foi enviado um email diretamente para cada um dos entrevistados com a definição do artigo base da técnica e foi perguntado se o entrevistado utilizava ou não a mesma, o que gerou mais 3 códigos. Ao final, 292 códigos surgiram ao todo.

O agrupamento entre códigos e práticas gerou a Tabela T1, que mostra uma visão geral de cada relação entre prática e entrevistado, contendo a nota dada e os códigos que justificam a nota. Esta tabela contém 147 códigos.

De posse da tabela T1 foi gerada a Tabela T2, que contem uma visão reduzida da primeira, contendo apenas as notas de cada uma das práticas para cada um dos entrevistados com as colunas ordenadas de acordo com o \emph{Stairway to Heaven} descrito no artigo base. Com a Tabela T2, também foi criada uma nova Tabela T3 com o mesmo formato da primeira, mas com as colunas ordenadas pela utilização de cada prática em ordem decrescente. Esta tem como objetivo replicar a Tabela 1 do artigo base \cite{empiricalStudy2016}.

Por fim, para identificar padrões e abstrações nos códigos agrupados na Tabela T1, foi feito um trabalho de agrupamento semântico, gerando por fim a tabela T4 apresentando as novas super categorias geradas no processo de \emph{Thematic Analysis}. Neste, 17 super categorias foram geradas.

É importante salientar que ainda durante o processo de levantamento de códigos surgiram tópicos relevantes que não se relacionavam diretamente com as práticas descritas, mas que são perguntados pelo questionário e relevantes para o tema tratado. Surgiram então 2 novas colunas que agregavam códigos sobre as práticas de \emph{Code Review} e de testes automáticos. Para estas foram relacionados 29 códigos, e duas super categorias foram geradas.

\section{Participantes}

No total foram entrevistados 11 desenvolvedores (P1 a P11) de 7 empresas diferentes sediadas em Recife. Destes, 3 eram mulheres. Pode-se ver a distribuição dos participantes entre gênero na \figref{genero}. Dentro desse grupo, 9 trabalhavam com aplicações Web, enquanto 1 trabalhava com sistemas embarcados e o último, com jogos.

\begin{figure}[ht]
\begin{center}
\includegraphics[scale=0.5]{demographics-genero.png}
\end{center}
\caption[Distribuição dos Participantes por gênero]{
    Distribuição dos Participantes por gênero
}\label{genero}
\end{figure}

A escolha dos entrevistados foi feita baseado na rede de conhecidos dos entrevistadores, com o propósito de agregar pessoas de que trabalhavam em empresas de tamanhos distintos para garantir uma variedade de parâmetros envolvidos. Como é possível perceber na \figref{tamanho_empresa}, com relação a esse aspecto a amostra está bem distribuída. 


\begin{figure}[ht]
\begin{center}
\includegraphics[scale=0.5]{demographics-tamanh-das-empresas.png}
\end{center}
\caption[Distribuição dos Participantes por tamanho da empresa]{
    Distribuição dos Participantes por tamanho da empresa.
}\label{tamanho_empresa}
\end{figure}




\chapter{Resultados}

resultados

\section{Primeiro tópico}

bla bla bla


\section{Conclusão}

Com a realização deste estudo, percebeu-se dentro da amostra a presença de pelo menos alguma das técnicas de CI/CD presentes no cotidiano das empresas Recifenses. Com a análise da entrevista qualitativa, foi possível entender um pouco melhor como tais técnicas foram importadas e também algumas das razões do porque foram adotadas. Ainda que a amostra não siga fielmente o \emph{Stairway to Heaven} proposto no estudo original \cite{empiricalStudy2016}, observou-se certa congruência no que diz respeito ao resultado de ambos os estudos.

Além disso, percebeu-se que testes A/B não eram utilizados, além de que \emph{trunk based development} se mostrou pouco utilizado, visto que a maioria entrevistada relatou que a integração do código de uma nova funcionalidade para a \emph{branch} principal de desenvolvimento era feita apenas no final da \emph{Sprint}.


%%
%% The acknowledgments section is defined using the "acks" environment
%% (and NOT an unnumbered section). This ensures the proper
%% identification of the section in the article metadata, and the
%% consistent spelling of the heading.
% \begin{acks}
% TODO escrever isso melhor
% \end{acks}

%%
%% The next two lines define the bibliography style to be used, and
%% the bibliography file.
\bibliographystyle{ACM-Reference-Format}
\bibliography{biblio}

%%
%% If your work has an appendix, this is the place to put it.
\appendix

\end{document}
\endinput
%%
%% End of file
