\begin{abstract}

As práticas de integração (CI), entrega (CDE) e implantação (CD) já estão presentes no cotidiano de grandes empresas de todo o mundo. Isto é devido, entre outras coisas, aos comprovados benefícios que a utilização destas técnicas trazem para a equipe e para o produto em desenvolvimento. Mesmo assim, poucos são os estudos que investigam o estado da arte destas práticas nas empresas, principalmente em ecossistemas específicos de software. Assim, o presente trabalho objetiva entender como as técnicas de integração, entrega e implantação contínuas foram importadas para as empresas recifenses, bem como identificar princípios e práticas adjacentes que governam a adoção destas técnicas. Para tanto, foi realizada uma pesquisa qualitativa por meio de entrevistas com 11 desenvolvedores de software de empresas de tecnologia sediadas na cidade do Recife. Foi descoberto que a amostra não segue o \emph{Stairway to Heaven}, teoria que sugere que a adoção deve seguir uma sequência específica de práticas. Não obstante, a maioria entrevistada integra o código de uma nova funcionalidade para a \emph{branch} principal apenas no final da \emph{Sprint}, não diariamente, o que não é consistente com as definições mais comuns de integração contínua. Além disso, a prática de Testes A/B não foi encontrada em nenhum dos times entrevistados, em razão da baixa quantidade de usuários ativa ou por não se aplicar ao contexto da aplicação.
\end{abstract}
% Palavras-chave do resumo em Português
\keywords{Integração Contínua, Implantação Contínua, Desenvolvimento Colaborativo, Engenharia de Software}
