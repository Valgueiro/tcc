\chapter{Fundamentação Teórica}

Este capítulo tem como objetivo apresentar conceitos importantes para a
construção desta pesquisa e entendimento dos resultados obtidos. Assim, o capítulo abordará conceitos de metodologias ágeis, DevOps, técnicas de integração, entrega e \emph{deployment} contínuous e práticas relacionadas. Por fim, serão abordados alguns dos trabalhos relacionados a este presentes na literatura.


\section{Metodologias Ágeis}

A citação a seguir retirada da dissertação \cite{experienciaComAgil} define bem os como os métodos ágeis surgiram.

    
\begin{quotation}[]{Dairton Luiz Bassi Filho}
    "Durante a evolução dos processos de Engenharia de Software, a
indústria se baseou nos métodos tradicionais de desenvolvimento de
software, que definiram por muitos anos os padrões para criação de
software nos meios acadêmico e empresarial. Porém, percebendo que a
indústria apresentava um grande número de casos de fracasso, alguns
líderes experientes adotaram modos de trabalho que se opunham aos
principais conceitos das metodologias tradicionais. Aos poucos, foram
percebendo que suas formas de trabalho, apesar de não seguirem os
padrões no mercado, eram bastante eficientes. Aplicando-as em vários 
projetos, elas foram aprimoradas e, em alguns casos, chegaram a se
transformar em novas metodologias de desenvolvimento de software.
Essas metodologias passaram a ser chamadas de leves por não
utilizarem as formalidades que caracterizavam os processos tradicionais
e por evitarem a burocracia imposta pela utilização excessiva de
documentos. Com o tempo, algumas delas ganharam destaque nos
ambientes empresarial e acadêmico, gerando grandes debates,
principalmente relacionados à confiabilidade dos processos e à
qualidade do software.
\end{quotation}

É notório que a antecipação total de requisitos, como propunha o modelo de Cascata antigo, era a fórmula perfeita para a falha do projeto, como comenta \cite{agileSoftwareDevelopment}. Principalmente em ambientes de desenvolvimento de software, a metodologia mais interessante deve se basear em entregas parciais rápidas, com o objetivo de encontrar erros em fases iniciais e gerar um produto final mais de acordo com a expectativa do cliente. A ideia, como comenta o artigo, é reduzir o custo de mudanças durante o todo o desenvolvimento do projeto. As metodologias ágeis surgiram então com o objetivo de suprir essas necessidades.

No geral, é possível perceber que as metodologias ágeis conseguiram encantar as empresas de desenvolvimento de software principalmente por dar a habilidade de gerenciar mudanças de prioridades, ajudar no alinhamento entre as equipes de mercado e de tecnologia e aumentar a velocidade de entregas, de acordo com \cite{stateAgileReport2020}. Pesquisas como esta demonstram o sucesso que esse novo paradigma obteve no contexto de produção de aplicações de T.I. 

\subsection{Scrum}
Uma das metodologias ágeis que apresenta uma grande quantidade de usuários é o SCRUM \cite{scrumBook}. Ele é um framework de gerenciamento de projetos que estabelece uma série de regras e cerimônias ao processo com o objetivo de garantir entregas interativas e incrementais. Mesmo tendo sido criado em 1993 por Jeff Suntherland, a metodologia é - entre as ágeis - a mais utilizada atualmente: de acordo com a pesquisa da \emph{digital.ia} de 2020 \cite{stateAgileReport2020}, 58\% das equipes utilizam o framework na sua organização.

O scrum consiste em...
* daily
* retrospectiva
* entregas parciais
* Scrum master e PO (??)


\section{Devops}
 
    * se tornou uma buzzword atualmente
    * baseado na Wikpedia, devops é um conjunto de práticas que combinam o desenvolvimento de software e as operações. A  ideia é diminuir o os silos entre as duas equipes.


\subsection{Integração Contínua}

\subsection{Entrega e \emph{Deployment} Contínuo}
* continuous integration
* continuous delivery
    * companias como flickrs postaram no passado em seus blogs a quantidade de "deploys" que eram feitos por dia/semana.
    * a ideia é que cada mudança que cada desenvolvedor faz é mandada diretamente para o ambiente de produção.
    As features podem não estar devidamente aparentes para os consumidores, através de feature toggles.
    * isso talvez não seja para todas as empresas, visto que organizações podem ter requisitos e políticas que necessitem de estágios manuais de validação
    * diferença entre continuous delivery e deployment. Delivery nãos ignifica que o código está indo sempre para produção, mas sim que ele está sempre no estado pronto para produção. 
    * talvez continuous deployment seja apenas uma opção, mas que não seja viável para a sua aplicação estar deployando sempre.
    * 

\subsection{Entregas Parciais}

* code review
* testes automáticos

\section{Trabalhos Relacionados}
